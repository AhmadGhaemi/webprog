%!TEX encoding = UTF-8 Unicode
%!TEX root = ../compendium1.tex

%--- HTTP -------------------------------------
\chapter{Network} \label{chapter:network}
This lecture covers topics related to communication between web applications and servers on the internet. 
The internet are commonly described with a four layered architecture of protocols, where each layer adds more functionality. The lowest layer, the \emph{link layer}, deals with communication on a local network and includes details on how the digital bits are converted to and from the analog signal sent over the transmission channel. Example of protocols at this layer are ethernet and the WiFi Standards 802.11a/b/g/n. The layers above add more functionality. The next layer is the \emph{Internet layer}. This layer makes it possible for packages to find it way around the globe. This is the responsibility of the IP-protocol. It also defines the IP-address and describes how the addresses are assigned to nodes on the internet. Each node on the internet have it own unique IP-address. Above this we have the transport layer and the TCP and UDP protocols. TCP is short for Transmission Control Protocol and provides flow-control, connection establishment, and reliable transmission of data. This means that TCP ensures a reliable stream of data. When data are lost, they will be retransmitted, so when using TCP, you do not need to bother about dividing your data into packages, you just write to a stream. TCP makes sure the receiver eventually get the same stream of data. For some applications, there is no value of data that arrives late. Then UDP is an alternative. UDP delivers packages, but some may be lost on the way and the packages may arrive in a different order than they were sent. On the top of the internet protocol stack you find the \emph{application layer}. This layer is the focus of this course. This is the home of protocols describing how applications communicate.

\section{URI and URL} \label{section:url}
A Uniform Resource Identifier (URI) identifies a resource, for example the ISBN number of a book. A Uniform Resource Locator (URL) is the location of a resource, i.e a web address. Both have the the same syntax:
\begin{Code}
URI = scheme:[//authority]path[?query][#fragment]
authority = [userinfo@]host[:port]
\end{Code}
Web applications are mainly interested in fetching and posting data, so this text will focus on URLs. Here is one example:
\begin{Code}
https://cs.lth.se/edaf90/?firstname=per#url
\end{Code}
The by far most commonly used protocol is \texttt{https}, but \texttt{http} and \texttt{ftp} are also common.  A url can contain any protocol, for example Secure Shell (SSH) or Network Time Protocol (NTP). However you can of course only use protocols your network stack knows how to handle. For web applications in a web browser, this means https/http which is supported by, \code{window.fetch}. 

The authority part of the example url is the DNS name of the server. An IP-number can also be used. If a port is not given, the default for the protocol is used, 443 for https and 80 for http. In the labs you use port 3000, which is used by the node.js development http server.

The path part of a url commonly identifies a file. If the path is a directory on the servers file system, the server usually responds with a default file from that directory, i.e. \texttt{index.html} or \texttt{index.php}. This behaviour is part of the server configuration, so do not rely on it if you do not have control over the server. 

The query part of a url is commonly generated as part of a http form submit. Here is an example:
\begin{Code}
<form action="/view/" method="get">
  Enter item id:<br>
  <input type="text" name="item">
</form>
\end{Code}
When the user submits the form, the browser send a request for (this can be interrupted using JavaScript, as you do in the labs):
\begin{Code}
https://cs.lth.se/view?item=42
\end{Code}
If you set the metod to POST, the form data is submitted in the body of a http POST request. POST is the only alternative if you want to upload files directly from a html form.

The last part of a url, the framgent, makes it possible to refer to a specific element in a html file. The fragment is the id of the referred element, for example:
\begin{Code}
<h2 id="url">URI and URL</h2>
...
<a href="https://cs.lth.se/edaf90/#url">
\end{Code}

A few years ago, the path in a url referred always referred to a file on the server, and any additional information the server needed to respond to the http request was passed in the query part of the URL. Let's look at an example. A web shop have a page for viewing items. Using the traditional approach, the URL to item 42 is:
\begin{Code}
https://my-shop.com/view?item=42
\end{Code}
This looks like a form submission, and some customers might hesitate to click on such link. Also, thinking of the URL as a pointer to a recourse, you want the path to refer to the item, not the code that visualises the data. Today another way to structure the URLs has become popular:
\begin{Code}
https://my-shop.com/view/2
\end{Code}
This looks like a normal link. It is still a URL. The only change is that the path no longer identifies the server side script. The server needs to do some pattern matching to figure out that the script is \texttt{view} and \texttt{42} is a parameter to the script. Not all web servers support this, for example apache can not do this out of the box (you can get the functionality using a few rewrite rules). An alternative is to use yet another structure for the same information:
\begin{Code}
https://my-shop.com/#view/2
\end{Code}
The part befor the fragment (\#) identifies the script. The script needs to interpret the ``view/2'' part. The router in react support the two last alternatives: \code{BrowserRouter} handles the \texttt{/view/2} structure, and \code{HashRouter} supports the \texttt{\#view/2} alternative.

\subsection{URL encoding} \label{section:http}
URLs are text strings, specifically a sequence of 7 bit ASCII characters. 7 bit ASCII can not represent many of the characters in latin, asian, or arabic alphabets, such as åäö. Also, some characters, for example \texttt{/\#?} have special meaning in a url. Clearly, there is a need to represent these characters in URLs, especially when a URL containing the submit of a html form. This is done by URL-encoding, also known as percent-encoding. The principle is to represent any non safe character with a three character sequence, \texttt{\%nn}, where \texttt{nn} is the hexadecimal value of the 8 bits of the character, for example  ``ett två'' is encoded as ``ett\%20tv\%C3\%A5'' (space is ASCII character 20, ``å'' is a two byte sequence in utf-8: 0xC3 and 0xA5). The safe characters, those that do not need percent-encoding, are the letters \texttt{a} to \texttt{z}, \texttt{A} to \texttt{Z}, numbers \texttt{0} to \texttt{9}, and \texttt{- \_ . \textasciitilde} (minus, underscore, dot and tilde). According to the latest standard, utf-8 encoding should be used for url-encoding. However, in the context of html form submission, web browsers will use the character encoding of the web page. In JavaScript, url-encodeing is handled by the function \code{encodeURIComponent()}.
	
%--- HTTP -------------------------------------
\section{HTTP} \label{section:http}
HTTP is the protocol used for almost all communication between web browsers and servers. It is text based, simple and designed to be human readable. The communication is always initialised from the client, the server only responds to requests. The client sends a command to the server, for example a requests for a \emph{resource}. The resource is commonly a file, or a web page. The client may also provide additional information  in the request header, for example authentication details. The server execute the command and responds with the result, and possible extra information in the response header. Lets look at an example of how to fetch a page from \texttt{cs.lth.se}:
\begin{Code}
GET /edaf90/ HTTP/1.1
HOST: cs.lth.se

\end{Code}
The first line always has three parts: 1, the command \texttt{GET} 2, the resource \texttt{/edaf90/} 3, protocol version \texttt{HTTP/1.1}.

The http protocol defines the following commands: \texttt{GET, HEAD, POST, PUT, DELETE, CONNECT, OPTIONS, TRACE, PATCH}. We discuss them in more details in the section about REST. More information can also be found at mozilla's documentation: \url{https://developer.mozilla.org/en-US/docs/Web/HTTP/Methods}.

The resource is the \texttt{ ``path[?query][\#fragment]''} part of a url. Do not put the authority part of the url here (server name). This information goes to the header section.
 
The optional header section starts at line 2. Each line contain one header field following the syntax: \texttt{header-name: value}. The value is the text until the next new line character. Leading white spaces are trimmed from the value.
A complete list of the http standard headers can be found here \url{https://developer.mozilla.org/en-US/docs/Web/HTTP/Headers}. You are free to use any name, so the protocol can be extended to the needs of your application. In earlier versions of the protocol custom headers should start with X-, but that was later dropped as some custom headers later became part of the standard. 

The header section and the request body are separated by one empty line. The body contains additional information sent to the server, for example the content of a http-form for \texttt{POST} requests. The body is normally empty for \texttt{GET} requests.

Here is the server response for the example above:
\begin{Code}
HTTP/1.1 200 OK
Date: Sat, 09 Feb 2019 15:51:26 GMT
Server: Apache/2.4.6 (Red Hat Enterprise Linux) OpenSSL/1.0.2k-fips PHP/7.0.27
Content-Language: sv
ETag: "7dc5c74268a1b0ea90ce76b44ebe3181"
Content-Type: text/html; charset=utf-8
x-url: /edaf90/
Age: 44
X-Cache: cached
X-Cache-Hits: 1
Content-Length: 38430
Connection: keep-alive
Accept-Ranges: bytes

<!DOCTYPE html>
... more text, a total of 38430 bytes
\end{Code}
The response have the same form as the request. The first line has three fields:\\ \texttt{protocol-version status-code human-readable-status}.
The protocol version is the latest the server supports, but never newer than what the client asked for in the request.

Status codes form groups:
\begin{itemize}
\item1xx (Informational): The request was received, continuing process.
\item2xx (Successful): The request was successfully received, understood, and accepted.
\item3xx (Redirection): The client needs to do additional actions in order to complete the request.
\item4xx (Client Error): The request contains bad syntax or cannot be fulfilled. The client must rephrase the request before any later attempts.
\item5xx (Server Error): The server failed to fulfill an apparently valid request. For example, because a database is down for maintenance. The client may try the same request again later.
\end{itemize}
The normal response is 200 --- OK. You probably also have seen 404 --- page not found. A list of status codes can be found at \url{https://developer.mozilla.org/en-US/docs/Web/HTTP/Status}, or simply google for ``http status code''. 

Http is text based, so any binary information, for example images, are always base64 encoded. Make sure you encode/decode any binary data before you send/reads it. Check the documentation of your http-library for the details. More information about base64 encoding can, for example be, found at wikipedia, \url{https://en.wikipedia.org/wiki/Base64}.

The latest version of the http standard is 2.0. This version keeps the same basic structure: command-header-body. Version 2.0 mainly focus on improving efficiency. The headers can be compressed, so they are no longer human readable in the network stream. The server is also allowed to send more data than was requested, for example when a client asks for a web page, it probably will ask for the images in that page later. The server can bundle all content of the page in the first response.

As a web application developer you do not need to bother about the inner details of the http protocol. All network accesses are made using libraries, both on the client and server side. They handle the formatting of the request, compression et.c. Base64 and url encoding may be handled different in different libraries, so check the documentation of your framework. To use the libraries you do need to know the basics: command, response status code, header and body as these are part of any http-api.

If you want to examen the inners of http-requests, you can use the \texttt{curl} command in a terminal.
\begin{Code}
curl -X GET http://cs.lth.se/edaf90
\end{Code}
%--- JSON -------------------------------------
\section{JSON} \label{section:json}
JSON is a popular text based file format commonly used in RESTful APIs. JSON is a subset of JavaScript. JSON always use unicode character encoding. There are five values:
\begin{itemize}
  \item a literal, one of \code{true, false} or \code{null}
  \item a number, for example \code{3.1415} or \code{42}
  \item a string, a sequence of unicode code points wrapped with quotation marks ", (U+0022). Note, you may not use singel quotation marks \code{'}, which JavaScript allows.
  \item an object, a unordered collection of comma separated name-value pairs wrapped with \texttt{\{ \}}. The name and the value are separated by :. The are no restrictions on the name. It may contain any character, but it must be a string (wrapped with quotation marks). In javaScript a name only needs quotation marks if it contains special characters, like space.
 \item an array, an ordered list of comma separated values wrapped with \texttt{[ ]}. You can mix different types of values in the same array.
\end{itemize}
Objects and arrays can be nested. Example:
\begin{Code}
{
  "Salad + Pasta": {price: 10, foundation: true, gluten: true},
  "'Norsk fjordlax": {price: 30, protein: true},
  "Cashewnötter": {price: 5, extra: true, vegan: true},
  "Sojabönor": {price: 5, extra: true, vegan: true},
  "Tomat": {price: 5, extra: true, vegan: true},
  "Ägg": {price: 5, extra: true},
  "Ceasardressing": {price: 5, dressing: true, lactose: true},
  "info": ["open 8-17", {"delivery":true}, [null, true, false]]
}
\end{Code}
More info about JSON can be found at \url{https://www.json.org}

In JavaScript there is a global object called JSON. It contains several handy functions, like \code{parse()} and \code{stringify()}.
\begin{Code}
const str = '{ "Clavin": {"alter ego": "Spaceman Spiff"}}';
const myObj = JSON.parse(str);
console.log(JSON.stringify(myObj)); 
\end{Code}
%--- XML -------------------------------------
%\section{XML} \label{section:xml}

%--- REST -------------------------------------
\section{REST} \label{section:rest}
Modern web applications are complex and have more in common with desktop applications than html forms from the early days of the web. The web browser comes with basic support for fetching pages and submitting forms. Using these mechanisms, the entire page needs to be fetched and rendered whenever the application need to interact with the server. This is not enough for modern web applications. The application needs to have direct communication with the server to exchange information without freezing the user interface. Web browsers have support for this, see \ref{section:ajax}. These communication APIs makes it possible for a web application to send http requests to any server. Http i just text, so every application needs to add its own protocol on top of http. There are many pitfalls and issues you need to be aware of when designing an api for a distributed system, which all web applications inherently are. REST is a set of principles and guidelines to help you develop an api on top of http. Terminology: when following these principles, you say that you have a RESTful api, providing a REST service running on a REST server.

The core principles of Representational State Transfer (REST) are:
\begin{itemize}
\item Client–server architecture
\item Statelessness
\item Cacheability
\item Layered system
\item Uniform interface
\end{itemize}

\subsection{Client–server architecture}
The principle behind the client–server constraints is the separation of concerns. Separating the user interface concerns from the data storage concerns improves the portability of the user interface across multiple platforms. It also improves scalability by simplifying the server components. Perhaps most significant to the Web, however, is that the separation allows the components to evolve independently, thus supporting the Internet-scale requirement of multiple organizational domains

\subsection{Statelessness}
Each request from any client contains all the information necessary to service the request, and session state is held in the client.

This eliminates many potential buggs. Let's illustrate this with bugg caused by a stateful protocol. The api of a web shop have two operations \texttt{/view-item/} and \texttt{/order-last-viewed-item/}. The customer do this:
\begin{Code}
/view-item/13
/view-item/42
  Go to the previous page using the browser history.
  The user sees item 13 on the screen
/order-last-viewed-item
\end{Code}
Pages in the browser history are often rendered from the browser cache. The server can not know that the customer was viewing item 13 when placing the order, so an order of item 42 was made. The same problems occurs when a user opens the same application in different tabs. You will get a copy of the DOM in the new tab, and once the user starts to interact with the application, the DOMs in the tabs diverges. There is no way for a server to keep track of which tab is the origin of a request.

\subsection{Cacheability}
As on the World Wide Web, clients and intermediaries can cache responses. Responses must therefore, implicitly or explicitly, define themselves as cacheable or not to prevent clients from getting stale or inappropriate data. Use http headers to control the cacheability of each response.

\subsection{Layered system}
A client cannot ordinarily tell whether it is connected directly to the end server, or to an intermediary along the way. Intermediary servers can improve system scalability by enabling load balancing and by providing shared caches. They can also enforce security policies. Layered system comes for free when building ontop of http.

\subsection{Uniform interface}
This is all about making things simple, more on this in \ref{section:crud}. The four constraints for this uniform interface are:
\begin{itemize}
  \item Resource identification in requests
  \item Resource manipulation through representations -- When a client holds a representation of a resource, including any metadata attached, it has enough information to modify or delete the resource.
  \item Self-descriptive messages -- Each message includes enough information to describe how to process the message. For example, which parser to invoke can be specified by a media type.
  \item Hypermedia as the engine of application state
 \end{itemize}
 
 Part of the text above comes from wikipedia,\\ \url{https://en.wikipedia.org/wiki/Representational_state_transfer}

\subsection{REST in practice, CRUD} \label{section:crud}
The text above is written from a theoretical point of view, and in general terms, talking about resources and interfaces. In practice things get relay simple. Resources are usually rows in databases and the operations we want to do are Create, Read, Update, and Delete (CRUD). REST leverage on http, so we use the http commands POST, GET, PUT, and DELETE. The status code of the respons indicate if the operation was successful, for example 200-ok or 404-not found.

Where do we put the data? Well, this can be debated. There is no right and wrong to this question. Be consistent and adopt to the context of your system. A common approach is to follow the html way of doing things:
\begin{itemize}
  \item \emph{Resource identification} is part of the url:
  \begin{Code}
  /students/
  /students/42/
  \end{Code}
  The first request selects all students, and the second a specific student.
  \item \emph{Filtering} can be done using the query part:
  \begin{Code}
  /students/?name=Per
  \end{Code}
  \item \emph{Names:}  plural nouns are commonly used in URLs. Use lower case letters. If you have multi word names, replace space with minus sign, i.e. \texttt{piano-char}. Then things looks nice after url-encoding (you do not want \texttt{/piano\%20char/}). Avoid CamelCase since the path of the URLs are case insensitive. Avoid verbs in the path, the http command contains this information
  \item \emph{Http headers:} use it for cache control and authentication.
  \item \emph{Http body:} data returned from the server goes into the response body. Create and Update sends its data in the body, but Read and Delete only identifies a resource using the URL (empty body).
  \item{JSON:} use json format in the http body.
\end{itemize}
%--- GraphQL -------------------------------------
%\section{GraphQL} \label{section:graphql}


%--- AJAX -------------------------------------
%\section{AJAX} \label{section:ajax}




