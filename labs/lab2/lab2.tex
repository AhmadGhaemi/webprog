\documentclass[fleqn, article, a4paper]{memoir}
\usepackage[english]{../latex/selthcsexercise}

\usepackage[utf8]{inputenc}
% Utilities.
\usepackage{graphicx}
\usepackage{url}
\usepackage{soul}
\usepackage{varioref}
\usepackage{nameref}
\usepackage{microtype}

\newcommand{\scode}[1]{\texttt{\small#1}}
\newcommand{\FIXBEFORECODE}{\vspace{-0.5\baselineskip}}
\newcommand{\FIXAFTERCODE}{\vspace{-\baselineskip}}

%---------------------------------------------------------------
\newenvironment{Hemarbete}%
{\begin{Assignments}[H]}{\end{Assignments}}

\newenvironment{Kontrollfragor}%
{\begin{Assignments}[K]\tightlist}{\end{Assignments}}

\newenvironment{Datorarbete}%
{\begin{Assignments}[D]}{\end{Assignments}}

\newenvironment{DatorarbeteCont}%
{\begin{Assignments}[D]\setcounter{Ucount}{\theSavecount}}{\end{Assignments}}

\newenvironment{Deluppgifter}%
{\begin{enumerate}[a)]\firmlist}{\end{enumerate}}


\newcommand{\commandchar}[1]{\textsc{#1}}

% Section styles.
\setsecheadstyle{\huge\sffamily\bfseries\raggedright} 
\setsubsecheadstyle{\Large\sffamily\bfseries\raggedright} 
\setsubsubsecheadstyle{\normalsize\sffamily\bfseries\raggedright} 

\setsecnumformat{} % numrera inte laborationerna
\renewcommand{\thesection}{\arabic{section}} % för referenser till laborationerna
\renewcommand{\thefigure}{\arabic{figure}}

%*****************************************************************
\begin{document}
\courseinfo{Web Programming}{\the\year}
\maketitle
\thispagestyle{titlepage}
\vspace{-4cm}

\subsection*{Lab 2}

\n This lab is about the react and bootstrap, \emph{objectives}:

\begin{enumerate}\firmlist
\item Understanding how a web page can be styled using css classes.
\item Get experience with basic react usage: components and props.
\item Get some experience using html forms.
\end{enumerate}

\subsubsection*{Bootstrap}
Open the bootstrap documentation to get an overview of the different bootstrap components to choose from. The pages contains examples, so it is easy to copy the template code.
\\ \url{https://getbootstrap.com/docs/4.4/components/jumbotron/}

%\clearpage
\subsection*{Background}

In the first lab you created JavaScript code to manage custom made salads. In this lab you will create a web page where a user can compose and order salads.

On the course home page you find the instructions for creating a new react project, see \url{http://cs.lth.se/edaf90/labs/}.

If you do not intend to use bootstrap modals, skip the \code{npm install jquery popper.js} step. If you change your mind, you can always add them later.

In this lab we use ECMAscript modules, so download the ES6 variant of the inventory file:
\begin{Code}
> cd lab2
> curl -o src/inventory.ES6.js 
          http://fileadmin.cs.lth.se/cs/Education/EDAF90/labs/lab2/inventory.ES6.js
\end{Code}

\subsection*{Assignments}

\begin{Assignments}

\item Study the relevant material for lecture 3, see \url{http://cs.lth.se/edaf90/reding-instructions/lecture-3/}. 

\item If you are using the linux system at LTH, remember to run \code{initcs} to add \code{node} to the path.

\item To compose a salad we will need to know what it can contain. In \code{src/App.js} add:
\begin{Code}
import inventory from './inventory.ES6';
\end{Code}

\item Create a component for composing salads. Pass \code{inventory} to it using \code{props}. I suggest you name it ComposeSalad:
\begin{Code}
import React, { Component } from 'react';

class ComposeSalad extends Component {
  constructor(props) {
    super(props);
    this.state = {};
  }

  render() {
    const inventory = this.props.inventory;
    let foundations = Object.keys(inventory).filter(
      name => inventory[name].foundation
    );
    return (
      <div className="container">
        <ul>
          {foundations.map(name => <li key={name}>{name}</li>)}
        </ul>
      </div>
    );
  }
}

export default ComposeSalad;
\end{Code}

\noindent A few observations: 1, remember to export the component name, otherwise you can't instantiate it outside the file. 2, note how you can mix JavaScript and JSX in the \code{render()} function. 3, \code{key=\{name\}} helps react track witch part of the DOM to render when data changes, read about it in the react documentation. 4, \code{className="container"} is a bootstrap class that adds some styling to the page so it looks nicer. Style other html elements you add with bootstrap css classes. 4, JSX does not have comments, but you can use embedded JavaScript for that:
\begin{Code}
<span>
  {/* this part won't do anything right now */}
</span>
\end{Code}

\item Lets use the component, instantiate it in \code{App.js}:
\begin{Code}
import ComposeSalad from './ComposeSalad';

// add this line to the existing JSX in your render() function:
<ComposeSalad inventory={inventory} />
\end{Code}

\item In your \code{ComposeSalad} react component, add a html form for composing a salad, see \url{https://reactjs.org/docs/forms.html}.\\
\noindent some requirements:
\begin{itemize}
  \item To get familiar with html and css, you must use native html tags, e.g. \code{<input> and <select>}, and style the using \code{className}. Most real world applications use frameworks, such as ReactBootstrap, which encapsulate the html tags and styling in react components. You can use this approach in the project.
  \item You must use bound components to handle form state. In the project you can use any from handling frameworks you desire.
\end{itemize}
\noindent Some hints:
\begin{itemize}
  \item The \code{ComposeSalad} should only render the html form. If you want to use modals, place that code in a separate component, \code{ComposeSaladModal}, or in \code{App}. \code{ComposeSaladModal} is recommended since it makes your code more reusable. We will use a router later and then you should remove the modal.
  \item React is based on the \emph{model-view} design pattern. \code{ComposeSalad} is the view and \code{\{Salad: salad, inventory\}} is the model. \code{ComposeSalad} contains all functionality for viewing the model. \code{Salad} is not aware of how it is visualised. Do not put any view details, such as html/react elements, in this class. This makes your code portable. You can reuse the \code{Salad} class in an Angular or Vue.js application, or change the styling to material design, without modification.
  \item Remember to bind your callback functions:\\ \code{this.handleChange = this.handleChange.bind(this);} Read why you sometimes need to bind your callbacks here \url{https://reactjs.org/docs/handling-events.html}.
  \item Use checkboxes, see the bootstrap documentation on how to style them. The html elements to use are \code{<input type='checkbox'>} and \code{<label>}.
  \item For checkboxes, the state of the DOM-element is stored in the property named \code{checked} (for other \code{<input>} types, the DOM state is stored in the property \code{value}). Do not assign \code{undefined} to it. To avoid this, you can use the JavaScript short cut behaviour of \code{||} \\ \code{<input checked=(this.state['Tomat'] || false)>}.
  \item \code{<select>} and \code{<option>} might be good alternatives for selecting the foundation and dressing.
  \item Use iterations in JavaScript (\code{Array.map} is recommended), avoid hard coding each ingredient (your may not assume which ingredients are present in inventory, so the 'Tomato' part of the example above is not ok)
  \item It is a good idea to create additional react components, for example SaladCheckbox, and/or SaladSection (two instances, one for extras and one for proteins). You can pass bound functions to subcomponents if you prefer to keep the callback functions in \code{ComposeSalad}.
  \item \code{<imput>} elements have a \code{name} attribute. Use it to store which ingredient it represents. In your callback function it is available in \code{event.target.name}.
  \item You may assume correct input for now, we will add form validation in the next lab.
  \item When the form is submitted, create a new Salad object and pass it to the parent, i.e. \code{App}. \code{App} should only handle \code{Salad} objects and not bother about the internals of the \code{ComposeSalad}, i.e. creating the object from the the html form state.
  \item \code{onSubmit} is the correct event for catching form submission. Avoid \code{onClick} on the submit button, it will miss submissions done by pressing the enter key in the html form.
  \item Clear the form after a salad is ordered, so the customer can start on a new salad from scratch. Note, you can not remove properties from the state object, just change their value. There are two alternatives: set values to \code{false} to indicate that this option is not selected, or store the form state in an object stored in a state property, i.e. \code{this.setState(\{formValues: \{\}\})\};}.
  \item The default behaviour of form submission is to send a http GET request to the server. We do not want this since we handle the action internally in the app. Stop the default action by calling \code{event.preventDefault()}.
\end{itemize}

\item Store the salad order, i.e. a list of \code{Salad} objects, in the state of \code{App}.

\item Create a react component to view the salad order. The order should be an input to the component, as \code{inventory} is in \code{ComposeSalad}.

\item Add the \code{ViewOrder} component to \code{App}, i.e. \code{<ViewOrder order='{this.state.order'}>}. This demonstrates the declarative power of react. When the state changes all affected subcomponents will automatically be re-renderd. Remember to use \code{this.setState(newValues)} to update the state.
\newline
\newline
An order can contain several salads. Remember to set the \code{key} attribute in the repeated html/JSX element. Avoid using array index as key. This can break your application when a salad is removed from the list. This is explained in many blog posts, for example \url{https://medium.com/@robinpokorny/index-as-a-key-is-an-anti-pattern-e0349aece318}.
\\ \noindent \emph{Hint 1:} Introduce an \code{id} property in the \code{Salad} objects. This is not only useful for the \code{key} attribute. It is also needed in all situations where you need to refer to a specific salad from the html code, e.g. a remove or edit button in the order view. 
\\ \noindent \emph{Hint 2:} To generate a unique id, add a static \code{getUniqueId()} property to \code{Salad}. Implement it using a \code{nextId} counter. Where do you place the counter? Is it private so you know no one else can tamper with it, e.g. forgetting to increment it when using a value.

\item \emph{Optional assignment 1} Add a remove button to the list of salads in the \code{ViewOrder} component. Remember, \code{props} are read only. The original list is in the \code{App} component.

\item \emph{Optional assignment 2}  add functionality so the user can edit a previously created salad. Add an edit button to each row in the list of salads in the \code{ViewOrder} component. You modify the \code{ComposeSalad} component so it can be used for editing. This is a good use case for a modal wrapper around the \code{ComposeSalad} component. Note, do all editing on a copy of the salad object, else you can not cancel an edit. \emph{Hint 1: } pass the the salad to edit as a \code{prop} to \code{ComposeSalad}. To create a new salad, pass an empty object. The \code{Salad.id} property is a good indicator to distinguish between create and update. If it has a id, you edit, and if it is \code{undefined}, you create a new salad. \emph{Hint 2: } Do this assignment in two steps, first add the functionality to view the salad, then continue with changes needed to save the updated salad.

\item This is all for now. In the next lab we will introduce a router and move the \code{ComposeSalad} and \code{ViewOrder} to separate pages.

\end{Assignments}

%!TEX encoding = UTF-8 Unicode
%!TEX root = ../compendium.tex

\clearpage\null\thispagestyle{empty}
\vfill

{
\setlength{\parindent}{0pt}
\emph{Editor}: Per Andersson \\

%%  LIST OF CONTRIBUTORS to https://github.com/lunduniversity/introprog
%    Please contact bjorn.regnell@cs.lth.se if you think you should be
%    on this list, or make a pull request with an update of file briefly
%    describing your contribtion in the commit text.
%    This work is licenced under CC-BY-SA-4.0.
%!TEX encoding = UTF-8 Unicode
%!TEX root = compendium/compendium.tex
\hyphenation{}
\emph{Contributors} in alphabetical order:\\
Björn Regnell
\emph{Contributors} in alphabetical order: Per Andersson \\
\\ \newline

\emph{Home}: \url{https://cs.lth.se/edaf90} \newline

\emph{Repo}: \url{https://github.com/lunduniversity/webprog} \\ \newline

This compendium is on-going work. \\ \textbf{Contributions are welcome!} \\
\emph{Contact}: \url{per.andersson@cs.lth.se}
\\ \newline

~\\ \newline

You can use this work if you respect this \emph{LICENCE}: CC BY-SA 4.0 \\
\url{http://creativecommons.org/licenses/by-sa/4.0/} \\
Please do \emph{not} distribute your solutions to lab assignments and projects.
\\ \newline
Copyright \copyright~ 2015-\the\year. \\
Dept. of Computer Science, LTH, Lund University. Lund. Sweden.\\
}

\end{document}
