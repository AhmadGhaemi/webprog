\documentclass[fleqn, article, a4paper]{memoir}
\usepackage[english]{../latex/selthcsexercise}

\usepackage[utf8]{inputenc}
% Utilities.
\usepackage{graphicx}
\usepackage{url}
\usepackage{soul}
\usepackage{varioref}
\usepackage{nameref}
\usepackage{microtype}

\newcommand{\scode}[1]{\texttt{\small#1}}
\newcommand{\FIXBEFORECODE}{\vspace{-0.5\baselineskip}}
\newcommand{\FIXAFTERCODE}{\vspace{-\baselineskip}}

%---------------------------------------------------------------
\newenvironment{Hemarbete}%
{\begin{Assignments}[H]}{\end{Assignments}}

\newenvironment{Kontrollfragor}%
{\begin{Assignments}[K]\tightlist}{\end{Assignments}}

\newenvironment{Datorarbete}%
{\begin{Assignments}[D]}{\end{Assignments}}

\newenvironment{DatorarbeteCont}%
{\begin{Assignments}[D]\setcounter{Ucount}{\theSavecount}}{\end{Assignments}}

\newenvironment{Deluppgifter}%
{\begin{enumerate}[a)]\firmlist}{\end{enumerate}}


\newcommand{\commandchar}[1]{\textsc{#1}}

% Section styles.
\setsecheadstyle{\huge\sffamily\bfseries\raggedright} 
\setsubsecheadstyle{\Large\sffamily\bfseries\raggedright} 
\setsubsubsecheadstyle{\normalsize\sffamily\bfseries\raggedright} 

\setsecnumformat{} % numrera inte laborationerna
\renewcommand{\thesection}{\arabic{section}} % för referenser till laborationerna
\renewcommand{\thefigure}{\arabic{figure}}

%*****************************************************************
\begin{document}
\courseinfo{Web Programming}{\the\year}
\maketitle
\thispagestyle{titlepage}
\vspace{-4cm}

\subsection*{Lab 2}

\n This lab is about the react and bootstrap, \emph{objectives}:

\begin{enumerate}\firmlist
\item Understanding how a web page can be styled using css classes.
\item Get experience with basic react usage: components and props.
\item Get some experience using html forms.
\end{enumerate}
\noindent This lab will take significantly more time to finish compared to lab 1.
\subsubsection*{Bootstrap}
Open the bootstrap documentation to get an overview of the different bootstrap components to choose from. The pages contains examples, so it is easy to reuse the basic building blocks by copying the template code. Note, the examples uses HTML attribute names, but you must use the equivalent JSX names in an React component. Replace \code{class} and \code{for} in the examples with \code{className} and \code{htmlFor}, for example replace \code{class="btn btn-primary"} with \code{className="btn btn-primary"}
\\ \url{https://getbootstrap.com/docs/5.3/components/buttons/}

%\clearpage
\subsection*{Set up}

In the first lab you created JavaScript code to manage custom made salads. In this lab you will create a web page where a user can compose and order salads. On the course canvas page you find the instructions for creating a new react project. Follow this and replace \code{App.js} with the one in canvas. You should now have an app with a headline, a container box listing the extras, and a footer.

\subsection*{Assignments}

\begin{Assignments}

\item Study the relevant material for lecture 3 and 4.

\item You will create React components during the lab. You should use function components with the \code{useState} hook.

\item Create a component for composing salads (download the template file from Canvas). All source files are stored in \code{./src} directory of your project:
\begin{Code}
import { useState } from 'react';

function ComposeSalad(props) {
  let foundations = Object.keys(props.inventory)
                    .filter(name => props.inventory[name].foundation);
  const [foundation, setFoundation] = useState('Pasta'); 
  const [extra, setExtra] = useState({Bacon: true, Fetaost: true}); 

  return (
    <div className="container col-12">
      <div className="row h-200 p-5 bg-light border rounded-3">
        <h2>Välj bas</h2>
        {foundations.map(name => <div key={name} className="col-4">{name}</div>)}
      </div>
    </div>
  );
}
export default ComposeSalad;\end{Code}

\noindent A few observations:
\begin{itemize}
  \item Remember to export the component name, otherwise you can't instantiate it outside the file.
  \item Note the JSX code in the function, it looks like HTML with embedded JavaScript.
   \item \code{key=\{name\}} helps react track witch part of the DOM to render when data changes, read about it in the react documentation (Quick Start/Rendering lists).
   \item \code{className="container"} is a bootstrap class that adds some styling to the page so it looks nicer. Style other html elements using bootstrap css classes.
   \item JSX does not have comments, but you can use embedded JavaScript for that:
\begin{Code}
<span>  {/* this part won't do anything right now */}  </span>
\end{Code}
\end{itemize}

\noindent Next we will instantiate ComposeSalad in App. Make sure \code{src/App.js} imports everything you will use. These are my imports:
\begin{Code}
import 'bootstrap/dist/css/bootstrap.css';
import { useState } from 'react';
import inventory from './inventory.mjs';
import ComposeSalad from './ComposeSalad'
\end{Code}

\noindent Add a \code{ComposeSalad} component in App, in the returned JSX expression add:
\begin{Code}
<ComposeSalad inventory={inventory} />
\end{Code}

\noindent Open the java script console in the web browser and to check for output. Varnings are printed here when React finds some problems. Always have the java script console open to see messages from react. Test this by removing the \code{key=\{name\}} attribute. There should not be any warnings or errors when you have finished the lab.You need to reload the page to clear the console after fixing the problem.

\emph{Reflection question 1:} The render function must be a pure function of \code{props} and the component state, the values returned by \code{useState()}. What happens if the output of the render function is depending on other data that changes over time?

\emph{Reflection question 2:} In the code above, the \code{foundations} array is computed every time the component is rendered. The inventory changes very infrequent so you might think this is inefficient. Can you cache \code{foundations} so it is only computed when \code{props.inventory} changes? Hint, read about the second parameter to \code{useEffect}, ``Learn React''/``Escape Hatches''/``Synchronizing with Effects'', \url{https://react.dev/learn/synchronizing-with-effects}. Is it a good idea? Read You Might Not Need an Effect: \url{https://react.dev/learn/you-might-not-need-an-effect}

\item In your \code{ComposeSalad} react component, add a HTML form for composing a salad. This requires a fair amount of code and knowledge. Divide the work to small incremental steps, for example:
\begin{enumerate}
  \item Read and understand what should be stored in the component state, see \url{https://react.dev/learn/thinking-in-react#step-3-find-the-minimal-but-complete-representation-of-ui-state}.
  \item Read all requirements and hints bellow.
  \item The user must select: one foundation, one protein, two or more extras, and one dressing. The \code{Salad} class from lab 1 is not suitable to work with here. Instead use a state variable for each of the selections, as indicated by the provided \code{ComposeSalad} skeleton.
  
We have two states entities related to salads: 1, The compose salad form state, which changes while the user is composing a salad. This state should be local to \code{ComposeSalad}. 2, The list of salads in the shopping basket. This state is created and modified by \code{ComposeSalad} and viewed by \code{ViewOrder} which we will write later. This state belongs to a common ancestor of the producer/viewer, the \code{App} component.
  
  \item You will use a html form with select and check boxes to compose the salad. The html form must view the \code{ComposeSalad} state. Not the other way around (never read the form state from the DOM). The React state is the ``single source of truth''
  
  \item We need to decide on how the html form fields should be stored in the component state. The \code{Salad} class is not suitable for holding the form state since it mixes the values from different form fields. Instead we use one state variable for each of the parts of the salad. Use strings for foundation, protein and dressing and an object containing the selected extras (or you can use the \code{Set} class from JavaScript standard library). Object property names are the names of the selected extras. Add and delete the properties as the user checks/unchecks the boxes).
  
  \item Start with the foundation part of a salad. 
  \begin{itemize}
    \item First view the component state for the foundation. Write the code to render a \code{<select>} and all \code{<option>} elements for the foundation. You wrote the code to render the \code{<option>} list in assignment 1 in lab 1. Use \code{<select value=\{...\}>} to enforce which option is selected. In React you set the \code{value} attribute in \code{<select>}. In vanilla html this is done by using the \code{selected} attribute in the \code{<option>} element (out of scope of the course).
    \item Make sure the component state controls which option is selected ("Pasta" is the initial state in the code above and should be selected in the browser). The user can not change this yet, since it is locked to the component state.
    \item Add an event handler, \code{onChange}, to the \code{<select>} element. The call back function can update the component state when the user selects a different foundation. The easiest way to check that the state is updated correctly is by inspect it using the React Developer Tools browser plugin.
  \end{itemize}
  \item When this works it is time to add some structure your code. You will use a similar \code{<select>} for foundation, protein, and dressing and it would be nice to reuse the code. Also, having many \code{<div>}, \code{<select>}, and \code{<option>} elements in the same render funktion makes it really hard to see what belongs to which part of the GUI. It is good to divide large components to several smal, and use components for structures that are repeated, for example the selects for foundation, protein and dressing. This is what I introduced in my solution:
  \begin{Code}
  <MySaladSelect
      options={foundations}
      value={foundation}
      onChange={onFoundationChange}
  />
  \end{Code}
  \emph{Reflection question 3:} Should you move the foundation state to the \code{MySaladSelect} component above? To answer this you need to consider who is using this information.

\item Implement the remaining part of the compose salad form. Protein and dressing should be straight forward using the \code{<MySaladSelect>} component. Add one more component containing the checkboxes for selecting extras.

\item Requirements:
\begin{itemize}
  \item The form should limit the selection to one foundation, one protein, zero or more extras, and one dressing. Use the html element \code{<select value=\{stateValue\}>} for foundation, protein, and dressing. Use one \code{<input type=checkbox name={nameOfExtra} checked=\{stateValue\}>} for each of the extras the customer can choose.

For checkboxes, the state of the DOM-element is stored in the atribute \code{checked} (for other \code{<input>} types, the DOM state is stored in the property \code{value}). Do not assign \code{undefined} to it. To avoid this, you can use the JavaScript short circuit behaviour of \code{||} \code{<input checked=(extras['Tomat'] || false)>}, or do an explicit type conversion: \code{Boolean(extras['Tomat'])}.
  \item \code{<imput>} elements have a \code{name} attribute. For checkboxes you can use it to store which ingredient it represents. In your callback function it is available in \code{event.target.name}.

  \item You do not need to support portion size (gourmet salad).

  \item You may assume correct input for now, we will add form validation in the next lab.

  \item One learning outcome of this lab is for you to get familiar with html and css. Therefore you must use native html tags, e.g. \code{<input> and <select>}, and style them using \code{className}. Most real world applications use frameworks, such as ReactBootstrap, which encapsulate the html tags and styling in react components. You should use this approach in the project but not in the labs.
  \item You must not read the form DOM state, use event handlers and update the component state variables.
  \item Your code must be flexible. If the content of \code{inventory} changes, your form should reflect these changes. Use iterations in JavaScript (\code{Array.map} is recommended), avoid hard coding html elements for each ingredient (you may not assume which ingredients are present in inventory). We will start with an empty inventory in lab 4 and add ingredients by fetching them from a rest server. Your \code{ComposeSalad} must support this.

  \item React is based on the \emph{model-view} design pattern. \code{ComposeSalad} is the view and component state and \code{this.props} is the model. \code{ComposeSalad} contains all functionality for viewing the model. \code{Salad} is not aware of how it is visualised. Do not put any view details, such as html/react elements, in the \code{salad} class. This makes your data structures portable. You can reuse the \code{Salad} class in an Angular or Vue.js application.
\end{itemize}

  \item \emph{optional assignment:} add a ``Caesar Salad'' quick compose button. When the user clicks the button, the form is pre-filled with the selections for a Caesar sallad.

\end{enumerate}
\emph{Reflection question 4:} What triggers react to call the render function and update the DOM?
\\\emph{Reflection question 5:} When the user change the html form state (DOM), does this change the state of your component?
\\\emph{Reflection question 6:} What is the value of \code{this} in the event handling call-back functions?


\item Handle form submission. The salad in the form should be added to a shopping cart when the user submits the form. The shopping cart should be stores in the \code{App} component.
\begin{itemize}
  \item When the form is submitted, you must create a \code{Salad} object from assignment 1 to store it.
  \item The shopping cart is a list of salad objects, use an array if you did not do the optional task in lab 1.
  \item The list of salads must be stored in \code{Apps} state since it will be use by other components later. When the form is submitted, create a Salad object in the callback function of \code{ComposeSalad} and pass it to \code{App}. Remember, the user might want to compose several salads, so make sure to copy/create objects when needed.
  \item \code{onSubmit} is the correct event for catching form submission. Avoid \code{onClick} on the submit button, it will miss submissions done by pressing the enter key in the html form.
  \item Clear the form after a salad is ordered, so the customer can start composing a new salad from scratch.
  \item The default behaviour of form submission is to send a http GET request to the server. We do not want this since we handle the action internally in the app. Stop the default action by calling \code{event.preventDefault()}. If you forget this then the app will be reloaded and JavaScript/component state will be lost (submit will result in an empty shopping cart).
\end{itemize}
\emph{Reflection question 7:} How is the prototype chain affected when copying an object with \\\code{copy = \{...sourceObject\}}?

\item Create a react component, \code{ViewOrder}, to view the shopping cart. The shopping cart should be passed from \code{App} using props. Instansiate the \code{ViewOrder} component in \code{App}, i.e. \code{<ViewOrder shoppingCart=\{shoppingCart\}>}. This demonstrates the declarative power of react. When the state changes all affected subcomponents will automatically be re-renderd.
\newline
\newline
An order can contain several salads. Remember to set the \code{key} attribute in the repeated html/JSX element. Avoid using array index as key. This can break your application when a salad is removed from the list. This is explained in many blog posts, for example \url{https://medium.com/@robinpokorny/index-as-a-key-is-an-anti-pattern-e0349aece318}.
\\ \noindent \emph{Hint 1:} use the \code{uuid} property in the \code{Salad} objects as key.

\item \emph{Optional assignment 1:} Add a remove button to the list of salads in the \code{ViewOrder} component. Remember, \code{props} are read only. The original list is in the \code{App} component.

\item \emph{Optional assignment 2:}  Add functionality so the user can edit a previously created salad. Add an edit button to each row in the list of salads in the \code{ViewOrder} component. For conditional rendering of a component you can use any JavaScript condition, \code{if...then...else} or \code{\{editMode \&\& <ComposeSalad edit=\{saladToEdit\}/>\} }. You also need modify the \code{ComposeSalad} component so it can be used for editing. Use \code{props} to pass the salad to be edited. If \code{App} will not initialise this prop, so it will be \code{undefined}. Use this to determine if the \code{ComposeSalad} component is in create or edit mode when needed, for example the the text for the submit button (create/update). Note: do not update the salad object in the order until the update button is pressed. This will change the state of \code{App}. Make sure to copy objects when needed and call the right setState function.

The edit scenario is a good use case for a modal wrapper around the \code{ComposeSalad} component. For edit, a pop-up window will appear, and when done the user is back in the list of the salads.
\\ \emph{Hint: } Do this assignment in two steps, first add the functionality to view the salad, then continue with changes needed to save the updated salad.

\item This is all for now. Make sure you do not have any warnings from React. Open the console, reload the app, compose a salad and view it. In the next lab we will introduce a router and move the \code{ComposeSalad} and \code{ViewOrder} to separate pages.

\end{Assignments}

%!TEX encoding = UTF-8 Unicode
%!TEX root = ../compendium.tex

\clearpage\null\thispagestyle{empty}
\vfill

{
\setlength{\parindent}{0pt}
\emph{Editor}: Per Andersson \\

%%  LIST OF CONTRIBUTORS to https://github.com/lunduniversity/introprog
%    Please contact bjorn.regnell@cs.lth.se if you think you should be
%    on this list, or make a pull request with an update of file briefly
%    describing your contribtion in the commit text.
%    This work is licenced under CC-BY-SA-4.0.
%!TEX encoding = UTF-8 Unicode
%!TEX root = compendium/compendium.tex
\hyphenation{}
\emph{Contributors} in alphabetical order:\\
Björn Regnell
\emph{Contributors} in alphabetical order: Per Andersson \\
\\ \newline

\emph{Home}: \url{https://cs.lth.se/edaf90} \newline

\emph{Repo}: \url{https://github.com/lunduniversity/webprog} \\ \newline

This compendium is on-going work. \\ \textbf{Contributions are welcome!} \\
\emph{Contact}: \url{per.andersson@cs.lth.se}
\\ \newline

~\\ \newline

You can use this work if you respect this \emph{LICENCE}: CC BY-SA 4.0 \\
\url{http://creativecommons.org/licenses/by-sa/4.0/} \\
Please do \emph{not} distribute your solutions to lab assignments and projects.
\\ \newline
Copyright \copyright~ 2015-\the\year. \\
Dept. of Computer Science, LTH, Lund University. Lund. Sweden.\\
}

\end{document}
