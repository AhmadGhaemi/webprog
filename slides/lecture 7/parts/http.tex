%--- OSI model ------------------------------------------------------------------------------
\begin{frame}[fragile] \frametitle{OSI model}
The protocols of internet are layered
\begin{itemize}
  \item Physical Layer
  \item Data Link Layer
  \item Network Layer
  \item Transport Layer
  \item Session Layer
  \item Presentation Layer
  \item Application Layer
\end{itemize}
\end{frame}

%--- MAC Address ------------------------------------------------------------------------------
\begin{frame}[fragile]\frametitle{MAC Address}
MAC address (simplified)
\begin{itemize}
  \item unique for each network interface
  \item static, set when manufactured
  \item manufaturer prefix
  \item 6 bytes, 48 bits
  \item used by the Data Link Layer
  \item example: 01-23-45-67-89-AB
\end{itemize}
\end{frame}

%--- IP Address ------------------------------------------------------------------------------
\begin{frame}[fragile]\frametitle{IP Address}
IP address
\begin{itemize}
  \item unique for each internet node
  \item dynamic, set when the node is added to internet
  \item version 4 and 6 are used
  \item 32 bit, or 128 bits
  \item used by the Transport Layer
  \item example: IPv4: 172.16.254.1, IPv6: 2001:db8:0:1234:0:567:8:1
\end{itemize}
Domain Name System (DNS)
\end{frame}

%--- Domain Name System (DNS)  ------------------------------------------------------------------------------
\begin{frame}[fragile]\frametitle{Domain Name System (DNS)}
Domain Name System (DNS)
\begin{itemize}
  \item distributed database
  \item mapps host name to IP-number
  \item uses the DNS protocol
\end{itemize}
\end{frame}


%--- URL and URI ------------------------------------------------------------------------------
\begin{frame}[fragile]\frametitle{URI and URL}
\color{structure}
\begin{itemize}\color{structure}
  \item URI --- identifies a resource
  \item URL --- identifies a location
  \item URL is a subset of URI
  \item URI = scheme:[//authority]path[?query][\#fragment]
  \item authority = [userinfo@]host[:port]
\end{itemize}
\vspace{5mm}
\begin{CodeBox}{}
<h2 id="url">URI and URL</h2>
...
<a href="https://cs.lth.se/edaf90/#url">
\end{CodeBox}
\end{frame}

%--- URL encoding ------------------------------------------------------------------------------
\begin{frame}[fragile]\frametitle{URL encoding}
URL is based on 7 bit ASCII
\begin{itemize}
  \item safe characters: a-z, A-Z, 0-9, \texttt{- \_ . \textasciitilde}
  \item all others must be percent-encoding
  \item \texttt{\%nn}, nn is 8 bit hexadecimal value
  \item percent-encoding uses utf-8
  \item \texttt{https://cs.lth.se/edaf90/lecture\%20notes}
\end{itemize}
\end{frame}

%--- HTTP ------------------------------------------------------------------------------
\section{HTTP}
%--- HTTP ------------------------------------------------------------------------------
\begin{frame}[fragile]\frametitle{HTTP}
\color{structure}
HTTP Request
\begin{itemize}\color{structure}
  \item text based - human readable
  \item request line
  \item header lines
  \item body
\end{itemize}
\texttt{
\noindent GET /edaf90/ HTTP/1.1\\
HOST: cs.lth.se
}
\end{frame}

%--- HTTP Commands------------------------------------------------------------------------------
\begin{frame}[fragile]\frametitle{HTTP Commands}
\color{structure}
Commands
\begin{itemize}\color{structure}
\item GET
\item HEAD
\item POST
\item PUT
\item DELETE
\item CONNECT
\item OPTIONS
\item TRACE
\item PATCH
\end{itemize}
\end{frame}

%--- Response ------------------------------------------------------------------------------
\begin{frame}[fragile]\frametitle{Response}
\color{structure}
HTTP Response
\begin{itemize}\color{structure}
\item  Status line
\item  header lines
\item  body
\end{itemize}

\texttt{
\noindent 
HTTP/1.1 200 OK
\\Date: Sat, 09 Feb 2019 15:51:26 GMT
\\Server: Apache/2.4.6 (Red Hat Enterprise Linux) OpenSSL/1.0.2k-fips PHP/7.0.27
\\Content-Type: text/html; charset=utf-8
\\X-Cache: cached
\\Content-Length: 38430
\\~ 
\\<!DOCTYPE html>
\\... more text, a total of 38430 bytes
}
\end{frame}

%--- Status Code------------------------------------------------------------------------------
\begin{frame}[fragile]\frametitle{HTTP Status Code}
\color{structure}
Status codes
\begin{itemize}
  \item  1xx (Informational)
  \item  2xx (Successful)
  \item  3xx (Redirection)
  \item  4xx (Client Error)
  \item  5xx (Server Error)
\end{itemize}
\end{frame}

%--- REST ------------------------------------------------------------------------------
\section{REST}
%--- REST ------------------------------------------------------------------------------
\begin{frame}[fragile]\frametitle{REST}
RESTful APIs are commonly for accessing {\bf data} stored on a server/database.
\vspace{3mm}
\\Principles:
\begin{itemize}
  \item client–server architecture
  \item statelessness
  \item cacheability
  \item layered system
  \item uniform interface
\end{itemize}
\end{frame}

%--- Uniform Interface ------------------------------------------------------------------------------
\begin{frame}[fragile]\frametitle{Uniform Interface}
\begin{itemize}
 \item resource identification in requests
 \item resource manipulation through representations
 \item self-descriptive messages
 \item hypermedia as the engine of application state
\end{itemize}
\end{frame}

%--- REST ------------------------------------------------------------------------------
\begin{frame}[fragile]
\frametitle{REST}
Not a standard, variant exist.
\vspace{3mm}
\\Common assumptions about a RESTful api:
\begin{itemize}
  \item uses HTTP for communication
  \item \emph{Http command} defines the action
  \item \emph{Http response code} indicates success/failure of the action
  \item \emph{Resource identification} is part of the url
  \item \emph{Filtering} can be done using the query part
  \item \emph{Http headers} provide additional information, e.g. authentication tokens
  \item \emph{Http body} contains the data
  \item {json format} is common, but XML is also used
\end{itemize}
\end{frame}


%---------------------------------------------------------------------------------
\begin{frame}[fragile]
\frametitle{CRUD}
Commonly supported commands
\begin{itemize}
  \item Create --- POST (new data in request body)
  \item Read --- GET
  \item Update --- PUT (new data in request body)
  \item Delete --- DELETE
\end{itemize}
\vspace{5mm}
Data manipulation operations returns the updated object in the response body.
\end{frame}

%--- Naming Convents ------------------------------------------------------------------------------
\begin{frame}[fragile]
\frametitle{Naming Convents}
\color{structure}
\noindent Names:
\begin{itemize}\color{structure}
  \item the api should be self explained
  \item only nouns
  \item plural names for collections
  \item avoid verbs, the http command contains this information
  \item use lower case letters.
  \item avoid CamelCase since the path of the URLs are case insensitive.
  \item separates words using the minus sign,\\  \texttt{/piano-chair/} vs \texttt{/piano\%20chair/})
\end{itemize}
\end{frame}

%--- JSON ------------------------------------------------------------------------------
\section{JSON}
%--- JSON ------------------------------------------------------------------------------
\begin{frame}[fragile]\frametitle{JSON}
Property names:
\begin{itemize}
  \item  are strings, must use \code{"quotation marks"}
\end{itemize}
Values:
\begin{itemize}
  \item  a literal, one of \code{true, false} or \code{null}
  \item  a number, for example 3.1415 or 42
  \item  a string must be \code{"must use quotation marks"}
  \item  an object \code{\{\}}, comma separated \emph{name : value} - pairs
  \item  an array \code{[]}, comma separated values
\end{itemize}
\begin{CodeBox}{}
{"name": "Per", "teacher": true }
\end{CodeBox}
\end{frame}


%--- REST ------------------------------------------------------------------------------
\section{Ajax}
%--- REST ------------------------------------------------------------------------------
\begin{frame}[fragile]\frametitle{Ajax}
Asynchronous JavaScript And XML
\begin{itemize}
  \item dynamic webpages using XMLHttpRequest
  \begin{itemize}
    \item event based (call back functions)
  \end{itemize}
  \item avoid whole page refresh
  \item update parts of the DOM
\end{itemize}
Single page web application
\begin{itemize}
  \item JavaScript frameworks such as REACT and Angular
  \item no page refresh, for the entire application
  \item uses the fetch API
  \begin{itemize}
    \item Prototype based
  \end{itemize}
\end{itemize}
\end{frame}

%--- fetch ------------------------------------------------------------------------------
\section{fetch}
%--- fetch ------------------------------------------------------------------------------
\begin{frame}[fragile]
\frametitle{fetch}
\begin{CodeBox}{}
fetch('http://example.com/movies.json')
  .then((response) => {
    return response.json();
  })
  .then((myJson) => {
    console.log(myJson);
  });
\end{CodeBox}
\end{frame}

%--- fetch ------------------------------------------------------------------------------
\begin{frame}[fragile]
\frametitle{fetch}
\begin{CodeBox}{}
fetch(url, {
  method: 'POST',
  mode: 'cors',
  cache: 'no-cache',
  credentials: 'same-origin',
  headers: {
    'Content-Type': 'application/json'
  },
  redirect: 'follow',
  referrerPolicy: 'no-referrer',
  body: JSON.stringify( { answer: 42 } )
});
\end{CodeBox}
\end{frame}

%--- Response ------------------------------------------------------------------------------
\begin{frame}[fragile]
\frametitle{Response}
\begin{CodeBox}{}
fetch('flowers.jpg')
  .then((response) => {
    if (!response.ok) {
      throw new Error('Network response was not ok');
    }
    return response.blob();
  })
  .then((myBlob) => {
    myImage.src = URL.createObjectURL(myBlob);
  })
  .catch((error) => {
    console.error('There has been a problem with your fetch operation:', error);
  });
  \end{CodeBox}
\end{frame}
