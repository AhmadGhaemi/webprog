%--- react------------------------------------------------------------------------------
\begin{frame}[fragile] \frametitle{React}
\begin{CodeBox}{}
const root = ReactDOM.createRoot(
  document.getElementById('root')
);
root.render(<h1>Hello, world!</h1>);
\end{CodeBox}

\vspace{8mm}
\begin{itemize}
  \item templates are JSX expressions
  \item JSX is embedded in the JavaScript code
  \item babel translates JSX to JavaScript code - builds a tree of React objects
  \item React injects the template object tree into the DOM
  \item rebuilt when a change happens
\end{itemize}

\end{frame}

%--- JSX ------------------------------------------------------------------------------
\section{JSX}
%--- JSX------------------------------------------------------------------------------
\begin{frame}[fragile] \frametitle{JSX}
\begin{itemize}
  \item any ``html tag'' is a JSX expression
  \item all tags must be closed, xml syntax works: \code{<img />}
  \item tags can be nested
  \item must be one root tag
  \item can use multiple lines
  \item use () to avoid automatic semicolon insertion!!!
  \item Babel compiles JSX down to \code{React.createElement()} calls
\end{itemize}

\begin{CodeBox}{}
const element = (
<span>
  <h1>Hello, world!</h1>
  <p>Some more text...</p>
</span>);
\end{CodeBox}
\end{frame}

%--- JSX attributes------------------------------------------------------------------------------
\begin{frame}[fragile] \frametitle{JSX attributes}
\begin{itemize}
  \item JSX tags can have attributes
  \item React DOM uses camelCase
  \item html: \code{class}, JSX: \code{className}
  \item html: \code{for}, JSX: \code{htmlFor}
  \item mapps to html attributes when injected to the DOM
\end{itemize}

\vspace{8mm}
\begin{CodeBox}{}
const element1 = <div tabIndex="0"></div>;
const element2 = <img src="picture.jpeg" />;
\end{CodeBox}
\end{frame}

%--- Embedded JavaScript ------------------------------------------------------------------------------
\begin{frame}[fragile] \frametitle{Embedded JavaScript}
JSX can contain embedded JavaScript
\begin{itemize}
  \item syntax: \{ \emph{JavaScript expression} \}
  \item use in:
  \begin{itemize}
    \item attribute values
    \item tag content
  \end{itemize}
  \item the embedded JavaScript expression may evaluate to another JSX expression
\end{itemize}
\vspace{3mm}
\begin{CodeBox}{}
const name = 'Per';
const imgSrc = 'picture.jpeg';
function doSomeMagic(arg) { return <span>a frog</span> };
const element1 = <h1>Hello {name}</h1>;
const element2 = <img src={imgSrc} />;
const element3 = (<span>{element1}
  <p>You are transformed to {doSomeMagic(name)}</p></span>);
\end{CodeBox}
\end{frame}

%--- React Components ------------------------------------------------------------------------------
\section{React Components}
%--- React Components------------------------------------------------------------------------------
\begin{frame}[fragile] \frametitle{Rect Components}
react components
\begin{itemize}
  \item JavaScript object:
  \begin{itemize}
    \item mechanism to render itself
    \item state
    \item function or class
  \end{itemize}
  \item a named template
  \item used in JSX expressions
  \item Model-View design pattern
\end{itemize}
\end{frame}

%--- Component example------------------------------------------------------------------------------
\begin{frame}[fragile] \frametitle{Component Example}
\begin{CodeBox}{}
  function HelloWorld(){
    return (
      <span>
        <h1>Hello</h1>
        <p>EDAF90 - web programming</p>
      </span>
    );
  }
\end{CodeBox}
\end{frame}
%--- Attributes------------------------------------------------------------------------------
\begin{frame}[fragile] \frametitle{Attributes - \code{props}}
a parent can pass data to a child
\begin{itemize}
  \item parent set the value using attributes in JSX
  \item the values are passed as a \code{props}
\end{itemize}
\end{frame}

%--- Properties Example ------------------------------------------------------------------------------
\begin{frame}[fragile] \frametitle{Properties Example}

\begin{CodeBox}{}
function HelloWorld(props) {
  return (
    <span>
      <h1>Hello World</h1>
      <p>{props.message}</p>
    </span>
  );
}

const element = (<HelloWorld message="web programming"/>);
\end{CodeBox}
\end{frame}

%--- Child Parent Communication ------------------------------------------------------------------------------
\begin{frame}[fragile] \frametitle{Child Parent Communication}
\begin{itemize}
  \item properties pass data down in the tree
  \item can also be used to pass data up the tree
  \begin{itemize}
    \item parent have a call-back function to update its state
    \item pass it as a property a child
    \item the child calls the parent call-back function
  \end{itemize}
\end{itemize}
\end{frame}


%--- State ------------------------------------------------------------------------------
\section{State}
%--- React Components------------------------------------------------------------------------------
\begin{frame}[fragile] \frametitle{Rect Class Components}
A React component can be a ES6 class:
  \begin{itemize}
    \item \code{extends React.Component}
    \item two data properties:
    \begin{itemize}
      \item \code{this.props}
      \item \code{this.state}
      \item immutable data structure, React updates them
    \end{itemize}
    \item \code{constructor(props)} should assign to \code{this.state}
    \item \code{render()} must be a pure function of \code{this.props} and \code{this.state}
  \end{itemize}
\end{frame}

%--- State Example ------------------------------------------------------------------------------
\begin{frame}[fragile] \frametitle{State Example}
\begin{CodeBox}{}
class Hello extends React.Component {
  constructor(props) {
    super(props);
    this.state = {name: props.name + ' Andersson'};
  }
  render() {
    return (
      <span>
        <h1>Hello {this.state.name}</h1>
        <p>{this.props.message}</p>
      </span>
  }
}
const element = <Hello name="Per" />;
\end{CodeBox}
\end{frame}

%--- State in class components ------------------------------------------------------------------------------
\begin{frame}[fragile] \frametitle{State in Class Components}
\begin{itemize}
  \item the state must be an immutable data structure
  \item use \code{setState()} to update state
  \item state is updated asynchronously (at some time in the future)
  \item state updates patches the old state:
  \begin{itemize}
    \item \code{nextState = \{ ...oldState, ...newState\}}
    \item you can not remove properties
    \item set undesired properties to \code{undefined}
  \end{itemize}
\end{itemize}
\end{frame}

%--- setState ------------------------------------------------------------------------------
\begin{frame}[fragile] \frametitle{setState}
Two variants:\\
new state is independent of old state
\begin{CodeBox}{\code{setState} takes a value:}
onReset() {
  this.setState( {count: 0} );
}
\end{CodeBox}

new state depends on old state
\begin{CodeBox}{\code{setState} takes a function:}
onClick() {
  this.setState(
    oldState => {count: oldState.count + 1}
);}
\end{CodeBox}

\end{frame}

%--- State in function components ------------------------------------------------------------------------------
\begin{frame}[fragile] \frametitle{State in Function Components}
\begin{itemize}
  \item must use hooks, introduced in 16.8.0
  \item component state is a sequence of values
  \item \code{useState} return the next value in the sequence
  \item the state values are returned in the same order in each render run
           \\do not call \code{useState} inside \code{if}, \code{while}
  \item \code{const [value, setState] = useState(initialState);}
  \item \code{setState} replace the state property value, no merge
  \item new state should not affect any ongoing rendering
\end{itemize}
\end{frame}

%--- State in function components ------------------------------------------------------------------------------
\begin{frame}[fragile] \frametitle{State in Function Components}
\begin{CodeBox}{useState hook}
function Example() {
  const [count, setCount] = useState(0);

  return (
    <div>
      <p>You clicked {count} times</p>
      <button onClick={() => setCount(n => n + 1)}>
        Click me
      </button>
    </div>
  );
}\end{CodeBox}
\end{frame}

%--- function vs class ------------------------------------------------------------------------------
\begin{frame}[fragile] \frametitle{Function vs Class Components}
Class Component
\begin{itemize}
  \item leverage of the full powers of the JavaScript language
  \item must understand:
  \begin{itemize}
    \item \code{this} --- how it gets a value
    \item prototype based inheritance
    \item use scopes/\code{bind()} for \code{this} in event handers
  \end{itemize}
\end{itemize}

Function Component
\begin{itemize}
  \item easy to use and understand
  \item everything is inside the scope of a function
  \item potential for ahead of time compilation in future
\end{itemize}

\end{frame}

%--- life cycle ------------------------------------------------------------------------------
\begin{frame}[fragile] \frametitle{Component Life Cycle --- class}
Mounting
\begin{itemize}
  \item \code{constructor()}
  \item \code{static getDerivedStateFromProps()}
  \item \code{render()}
  \item \code{componentDidMount()}
\end{itemize}
Updating
\begin{itemize}
  \item \code{static getDerivedStateFromProps()}
  \item \code{shouldComponentUpdate()}
  \item \code{render()}
  \item \code{getSnapshotBeforeUpdate()}
  \item \code{componentDidUpdate()}
\end{itemize}
\end{frame}

%--- life cycle ------------------------------------------------------------------------------
\begin{frame}[fragile] \frametitle{Component Life Cycle --- class}
Unmounting
\begin{itemize}
  \item \code{componentWillUnmount()}
\end{itemize}

Error Handling
\begin{itemize}
  \item \code{static getDerivedStateFromError()}
  \item \code{componentDidCatch()}
\end{itemize}

\end{frame}

%--- Render Elements------------------------------------------------------------------------------
\begin{frame}[fragile] \frametitle{Render Elements}
\code{ReactDOM.render()}
\begin{itemize}
  \item updates the DOM
  \item takes two parameters:
  \begin{itemize}
    \item a react element (JSX expression)
    \item a DOM element
  \end{itemize}
  \item updates the DOM if the element already was part of the DOM
  \item optimised, only updated the delta
\end{itemize}
\begin{CodeBox}{bootstrap the app}
ReactDOM.render(
   <App />,
  document.getElementById('root')
);
\end{CodeBox}
\end{frame}

%--- Event Handling ------------------------------------------------------------------------------
\section{Event Handling}
%--- Handling Events------------------------------------------------------------------------------
\begin{frame}[fragile] \frametitle{Handling Events}
\begin{itemize}
  \item event names are camelCase in JSX
  \item must use \code{preventDefault}, returning \code{false} do nothing
  \item pass a function: \code{onClick =\{myCallbackFunction\}}
  \item called with one argument, a synthetic event according to the W3C spec
\end{itemize}
\begin{CodeBox}{}
function ActionLink() {
  function handleClick(e) {
    e.preventDefault();
    console.log('The link was clicked.');
  }
  return (
    <a href="#" onClick={handleClick}>Click me</a>
  );
}
\end{CodeBox}
\end{frame}

%--- About this------------------------------------------------------------------------------
\begin{frame}[fragile] \frametitle{About this}
\begin{itemize}
  \item by default \code{this} is \code{undefined} in a callback function
  \item use \code{bind()} to bind it to the component object
  \item or arrow function:
\end{itemize}
\begin{CodeBox}{}
  constructor(props) {
    super(props);
    this.state = {count: 0};
    this.handleClick = this.setState(
      state => ({count: state.count + 1});
  )};
\end{CodeBox}
\end{frame}
%--- this example------------------------------------------------------------------------------
\begin{frame}[fragile] \frametitle{this example}
\begin{CodeBox}{}
class Counter extends React.Component {
  constructor(props) {
    super(props);
    this.state = {count: 0};
    this.handleClick = this.handleClick.bind(this);
  }
  handleClick() {
    this.setState(state => ({count: state.count + 1});
  }
  render() {
    return (
      <button onClick={this.handleClick}>
        {this.state.count}
      </button>
    );
  }
}
\end{CodeBox}
\end{frame}

%--- Lists and key ------------------------------------------------------------------------------
\section{Lists and key}
%--- Lists and key------------------------------------------------------------------------------
\begin{frame}[fragile] \frametitle{Lists and key}
\begin{itemize}
  \item embedded JavaScript may return a collection of react elements
  \item added in sequence to the DOM
  \item to help the render optimisation:
  \begin{itemize}
    \item each element should have a \code{key} property
    \item unique among siblings
    \item the value must be preserved over time
    \item avoid array index as key (changes when elements are deleted)
  \end{itemize}
\end{itemize}
\end{frame}

%--- Key example------------------------------------------------------------------------------
\begin{frame}[fragile] \frametitle{\code{key} example}

\begin{CodeBox}{}
function MyList(props) { return (
  <ul>
    {props.list.map(row => (
      <li key={row.id}>
        {row.text}
      </li>
    ))}
  </ul>);
}
\end{CodeBox}
\end{frame}

%--- Forms------------------------------------------------------------------------------
\begin{frame}[fragile] \frametitle{Forms}
\begin{itemize}
  \item DOM forms contain state
  \item hard for you JavaScript code to read that state
  \item use the \emph{controlled component} pattern
  \begin{itemize}
    \item the react component have the master state
    \item the DOM reflects the component state
    \item event listener (\code{onChange}): update the component state
  \end{itemize}
\end{itemize}
\vspace{5mm}
\begin{CodeBox}{}
<form onSubmit={this.handleSubmit}>
  <input type="text" 
        value={this.state.value}
        onChange={this.handleChange}
  />
</form>
\end{CodeBox}
\end{frame}

%--- Forms, select------------------------------------------------------------------------------
\begin{frame}[fragile] \frametitle{Form, select}
\begin{itemize}
  \item html, uses \code{selected} attribute in an \code{option} tag
  \item react uses a \code{value} attribute in the \code{select} tag
\end{itemize}
\vspace{3mm}
html
\begin{CodeBox}{}
<select>
  <option value="Coconut">Coconut</option>
  <option selected value="lime">Lime</option>
</select>\end{CodeBox}
\vspace{3mm}
react
\begin{CodeBox}{}
<select value={this.state.value} onChange={this.handler}>
  <option value="lime">Lime</option>
  <option value="coconut">Coconut</option>
</select>\end{CodeBox}
\end{frame}

%--- Forms, textarea------------------------------------------------------------------------------
\begin{frame}[fragile] \frametitle{Form, textarea}
\begin{itemize}
  \item html, value is the content
  \item react uses a \code{value} property
\end{itemize}
\vspace{3mm}
html
\begin{CodeBox}{}
<textarea>
  Hello there, some text in a text area
</textarea>
\end{CodeBox}
\vspace{3mm}
react
\begin{CodeBox}{}
const text = 'Hello there, some text in a text area';
const elem = (
  <textarea value={this.state.value}
               onChange={this.handleChange} />;
\end{CodeBox}
\end{frame}

%--- Forms, file input------------------------------------------------------------------------------
\begin{frame}[fragile] \frametitle{Form, file input}
\begin{CodeBox}{}
<input type="file"/>
\end{CodeBox}
\vspace{8mm}

\begin{itemize}
  \item is read-only
  \item must be uncontrolled
\end{itemize}
\end{frame}

%--- Forms call back------------------------------------------------------------------------------
\begin{frame}[fragile] \frametitle{Form, event handlers}
\begin{CodeBox}{}
class NameForm extends React.Component {
  constructor(props) {
    super(props);
    this.state = {value: ''};
    this.handleChange = this.handleChange.bind(this);
    this.handleSubmit = this.handleSubmit.bind(this);
  }

  handleChange(event) {
    this.setState({value: event.target.value});
  }
  handleSubmit(event) {
    alert('A name was submitted: ' + this.state.value);
    event.preventDefault();
  }
\end{CodeBox}
\end{frame}

%--- parametrised event handler------------------------------------------------------------------------------
\begin{frame}[fragile] \frametitle{Parametrised Event Handler}

\begin{CodeBox}{}
<input type="text" name="givenName" 
  value="this.state.givenName">
\end{CodeBox}
\begin{CodeBox}{}
handleInputChange(event) {
  const target = event.target;
  const value =
    target.type === 'checkbox' ?
        target.checked : target.value;
  const name = target.name;

  this.setState({
    [name]: value
  });
}
\end{CodeBox}
\end{frame}

%--- Form refs------------------------------------------------------------------------------
%\begin{frame}[fragile] \frametitle{Form refs}
%\begin{CodeBox}{}
%const input = React.createRef();
%function handleSubmit(event) {
%  alert('A name was submitted: ' + input.current.value);
%  event.preventDefault();
%}
%
%const form = (
%  <form onSubmit={this.handleSubmit}>
%    <label> Name:
%      <input type="text" ref={input} />
%    </label>
%    <input type="submit" value="Submit" />
%  </form>);
%\end{CodeBox}
%\end{frame}
