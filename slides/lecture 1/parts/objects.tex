%---Objects ---------------------------------------------------
\begin{frame}[fragile] \frametitle{Objects}
\begin{itemize}
  \item an object is a map: string $\rightarrow$ any value
  \item objects have properties - the keys of the map (attributes + methods in Java)
  \item properties can have any name, including reserved words and operations
  \item access properties using:
  \begin{itemize}
    \item dot notation: \code{myObj.prop}
    \item array index notation: \code{myObj['prop']}
  \end{itemize}
  \item add properties by writing to them \code{myObj.newProp = 'adding stuff';}
  \item remove properties by: \code{delete myObj.newProp}
\end{itemize}
\end{frame}

%---Objects, example 1 ---------------------------------------------------
\begin{frame}[fragile] \frametitle{Objects, example 1}
 \begin{CodeBox}{basic object}
let myObject = {
  givenName: 'Per',
  familyName: 'Andersson',
  selector: 'givenName',
  getValue: function () {
    return this[this.selector];
  }
}
 
console.log(myObject.getValue());
myObject.selector = 'familyName';
console.log(myObject.getValue());
 \end{CodeBox}
\end{frame}

%---Object Literals ---------------------------------------------------
\begin{frame}[fragile] \frametitle{Object Literals}
\begin{itemize}
  \item superset of JSON
  \item comma separated list of properties inside \code{\{ \}}
  \item a property is defined by: \code{property-name : value}
  \item name in plain text, quotes if needed 
  \item value is any JavaScript expression
  \item \code{\{a:a\}} is the same as \code{\{a\}}
\end{itemize}
\begin{CodeBox}{object literal}
let myObject = {
  five: 2 + 3,
  '5': 'five',
  '+' : 'plus',
  'null': 'not a good idea name'
 }
\end{CodeBox}
\end{frame}

%---Object Literals 2 ---------------------------------------------------
\begin{frame}[fragile] \frametitle{Object Literals}
\begin{itemize}
  \item object literals are cheap
  \item use them frequently
  \item they bring structure and readability to programs
\end{itemize}
\begin{CodeBox}{object literals}
let myPoints = [{a: 0, y: 0}, {x:10, y:15}];

function foo(a, b, c, d, e, f) {
 console.log('d = '+ d);
}
function bar(arg) {
 console.log('d = '+ arg.d);
}
\end{CodeBox}
\end{frame}

%--- Named Parameters ---------------------------------------------------
\begin{frame}[fragile] \frametitle{Named Parameters}
Remember, \code{foo} and \code{bar} prints parameter \code{d}.
\vspace{5mm}
\begin{CodeBox}{What is printed?}
foo(0, 0, 0, 0, 1, 1, 1);
bar({a: 0, b: 0, c: 0, d: 1, e: 1, f:1});
\end{CodeBox}
\vspace{10mm}
Did you notice thet foo have one extra parameter compared to the arguments list?
\\ Too few, or extra parameters do not raise errors in JavaScript. 
\end{frame}

%--- Constructor Functions ---------------------------------------------------
\begin{frame}[fragile] \frametitle{Constructor Functions}

\begin{itemize}
  \item like classes in Java: forms objects when they are created
  \item normal functions
  \item the intended usage differs
  \item by convention: use leading capital letter in name (similar to class names in Java)
  \item only call constructor functions using \code{new}:
  \begin{itemize}
    \item new creates an empty object and binds it to \code{this}
    \item the constructor function is called, adds properties to the object
    \item the return value of the constructor function will be the result of \code{new}
    \item remember: the default return value of functions called by \code{new} is \code{this}
  \end{itemize}
  \item combine with closure and you have the power
\end{itemize}
\end{frame}

%--- Constructor Functions  Example ---------------------------------------------------
\begin{frame}[fragile] \frametitle{Constructor Function Example}

\begin{columns}[onlytextwidth]
  \begin{column}{0.5\textwidth}
\begin{CodeBox}{class definition}
function Point(x, y) {
  this.x = x || 0;
  this.y = y || 0;
  this.getX = function() {
    return this.x;
  }
}
\end{CodeBox}
  \end{column}
  \begin{column}{0.5\textwidth}
\begin{CodeBox}{create instances}
let point1 = new Point(3, 6);
let point2 = new Point();
let point2 = new Point(5);
let point3 = 
  new Point(null, 5);
\end{CodeBox}
  \end{column}
\end{columns}%
\end{frame}

%---Undefined Names ---------------------------------------------------
\begin{frame}[fragile] \frametitle{Access to Undefined Names}
Variables:
\begin{itemize}
  \item read: throws ReferenceError
  \item write: creates the variable in the current scope
\end{itemize}
\vspace{5mm}
Properties:
\begin{itemize}
  \item read: evaluates to \code{undefined}
  \item write: adds the property to the object
\end{itemize}
\end{frame}

%--- Prototype based Inheritance ---------------------------------------------------
\begin{frame}[fragile] \frametitle{Prototype Based Inheritance}

\begin{itemize}
  \item all object inherit from another object
  \item objects forms a \emph{prototype chain}
  \item property name lookup follows the prototype chain
  \item you can access the prototype chain (but don't):
  \begin{itemize}
    \item \code{Object.getPrototypeOf(object)}
    \item \code{Object.setPrototypeOf(object, chain)}
  \end{itemize}
  \item the chain ends with \code{null}
\end{itemize}
\end{frame}

%--- Set up Prototype Chain ---------------------------------------------------
\begin{frame}[fragile] \frametitle{Set up Prototype Chain}
Setting up the prototype chain when a new object is created:
\begin{itemize}
  \item you can do it manually (but don't)
  \item \code{new} do the work for you
  \item all \code{Function} objects have a property \code{prototype}
  \item remember, constructor functions are instance of \code{Function}
  \item \code{new}:
  \begin{itemize}
    \item creates an empty object
    \item {\bf and} set its parent in the prototype chain to the \code{prototype} of the constructor function
  \end{itemize}
  \item all names in the \code{prototype} of the constructor function are now in the prototype chain of the new object
\end{itemize}
\end{frame}

%--- Set up Prototype Chain, example ---------------------------------------------------
\begin{frame}[fragile] \frametitle{example}
\begin{CodeBox}{prototype chain}
function Family(name) {
  this.givenName = name || 'Per';
}
Family.prototype.familyName = 'Andersson';
Family.prototype.toString = function() {
  return this.givenName + ' ' + this.familyName;
}

let me = new Family();
let sister = new Family('Anette');
console.log(me + ' and my sister ' + sister);
me.familyName = 'Andersson Nilsson';
console.log(me + ' and my sister ' + sister);
\end{CodeBox}
\end{frame}

%--- Property Name Lookup ---------------------------------------------------
\begin{frame}[fragile] \frametitle{Property Name Lookup}
Property read:
\begin{itemize}
  \item follows the prototype chain
  \item return the first value found
  \item return \code{undefined} if the end of the prototype chain is reached
\end{itemize}
\vspace{8mm}
Property write:
\begin{itemize}
  \item do not follows the prototype chain
  \item writes to the referenced object (left hand side of the dot)
  \item update if the name existed
  \item adds the property if the name did not exist
\end{itemize}
\end{frame}

%--- Inheritance ---------------------------------------------------
\begin{frame}[fragile] \frametitle{Inheritance}
\begin{itemize}
  \item the \code{prototype}s of the constructor functions form the prototype chain
  \item \code{Object.create()} creates an object with a given prototype chain
  \item use it to create the \code{prototype} of the constructor function
  \item explicit call the constructor of the superclass
\end{itemize}
\vspace{2mm}
\begin{CodeBox}{JuniorFamily extends Family}
function JuniorFamily(name) {
  Family.call(this);
}
JuniorFamily.prototype = Object.create(Family.prototype);
JuniorFamily.prototype.toString = function() {
  return this.givenName + ' ' + 
             this.familyName + '  jr.';
}
\end{CodeBox}
\end{frame}

%--- Class ---------------------------------------------------
\begin{frame}[fragile] \frametitle{Class}
\begin{itemize}
  \item \code{Class} was introduced in ECMAScript 2015
  \item syntactical sugar, set up the prototype chin as outlined above
  \item \code{static} will add the property to the constructor function object
\end{itemize}
\end{frame}
%--- Class Example ---------------------------------------------------
\begin{frame}[fragile] \frametitle{Class Example}
\begin{CodeBox}{}
class Family {
  constructor(name) {
    super();
    this.givenName = name || 'Per';
    Family.count = Family.count + 1;
  }
  toString() {
    return this.givenName + ' ' + this.familyName;
  }
  familyName = 'Andersson';
  static count = 0;
}
\end{CodeBox}
\end{frame}

%--- Class Extends---------------------------------------------------
\begin{frame}[fragile] \frametitle{Class Extends}
\begin{itemize}
  \item the constructor in a derived class must call \code{super()} before you can use \code{this}
  \item you can: \code{extend null}
\end{itemize}
\begin{CodeBox}{}
class JuniorFamily extends Family {
  constructor(name) {
    super(name);
  }
 toString() {
   return this.givenName + ' ' + 
              this.familyName + '  jr.';
  }
}
\end{CodeBox}
\end{frame}

%--- Standard Classes---------------------------------------------------
\begin{frame}[fragile] \frametitle{Standard Classes}
In JavaScript there are many standard classes. Some important: 
\begin{itemize}
  \item \code{Object} - default base class for all objects
  \item \code{Function extends Object} - base class for all functions
  \item \code{Array} - base class for array litterals
\end{itemize}
\end{frame}

%--- More to learn---------------------------------------------------
\begin{frame}[fragile] \frametitle{More to learn}

This is the basics of objects in JavaScript, but there are more under the surface\ldots
\vspace{8mm}
\begin{CodeBox}{}
Object.defineProperty(obj, "prop", {
    value: "test",
    writable: false
});
\end{CodeBox}
\vspace{8mm}
This is however out of scope for this course.
\end{frame}

%--- this---------------------------------------------------
\begin{frame}[fragile] \frametitle{this}
\begin{itemize}
  \item \code{this} is defined in all functions
  \item arrow functions, \code{this} from the enclosing scope is used
  \item its value depends on how the function is called:
  \begin{itemize}
    \item function call: \code{foo()} - the global object
    \item dot notation: \code{obj.foo()} - the object left of the dot
    \item explicit: \code{Function.prototype.call()}
    \item explicit: \code{Function.prototype.bind()} - creates a new function with a predefiend value for \code{this}
  \end{itemize}
\end{itemize}
\end{frame}

%--- Arrays---------------------------------------------------
\begin{frame}[fragile] \frametitle{Arrays}
\begin{itemize}
  \item variable size and type
  \item index must be number
  \item \code{myArray['per'] = 3;} - add a property to the array object
\end{itemize}
\end{frame}