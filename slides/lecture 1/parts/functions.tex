%---Functions ---------------------------------------------------
\begin{frame}[fragile] \frametitle{Functions}

\begin{itemize}
  \item functions are values - \code{Function} objects
  \begin{itemize}
    \item assign functions to variables
    \item use functions as arguments to other functions
  \end{itemize}
  \item call by value - like in Java (objects are references)
  \item default return value:
  \begin{itemize}
    \item \code{undefined}
    \item \code{this} in constructors
  \end{itemize}
  \item three ways to create functions:
  \begin{itemize}
    \item function declaration
    \item function expression
    \item \code{Function} constructor (not recommended)
  \end{itemize}
\end{itemize}

\end{frame}

%---Function Declaration ---------------------------------------------------
\begin{frame}[fragile] \frametitle{Function Declaration}

\begin{itemize}
  \item is a statement
  \item no need to use semicolon after a function declaration
  \item creates a \code{Function} object
  \item creates a variable with the function name
\end{itemize}
\begin{CodeBox}{function declaration}
function calcRectArea(width, height) {
  return width * height;
}

console.log(calcRectArea(5, 6));
\end{CodeBox}
\end{frame}

%---Function Expression ---------------------------------------------------
\begin{frame}[fragile] \frametitle{Function Expression}

\begin{itemize}
  \item is an expression
  \item creates a \code{Function} object
  \item the function name is optional, omitting it creates an anonymous function
  \item the name is stored in the \code{Function} object, can only be used inside the function
  \item you must store the value to use the function
\end{itemize}
\begin{CodeBox}{function expression}
let calcRectArea = function foo(width, height) {
  return width * height;
}

console.log(calcRectArea(5, 6));
\end{CodeBox}
\end{frame}

%---Default Parameters ---------------------------------------------------
\begin{frame}[fragile] \frametitle{Default Parameters}
\begin{itemize}
  \item function parameters default to \code{undefined}
  \item parameters can have other default values (ES2015)
  \item parameters are available to later default parameters
  \item default parameters are evaluated at call time
\end{itemize}

\begin{CodeBox}{rest parameters}
function multiply(a, b = 1) {
  return a * b;
}

function greet(name, 
                    greeting,
                    message = greeting + ' ' + name) {
    return [name, greeting, message];
}
\end{CodeBox}
\end{frame}

%---Rest Parameters ---------------------------------------------------
\begin{frame}[fragile] \frametitle{Rest Parameters}
\begin{itemize}
  \item must be the last named parameter
  \item all remaining arguments are wrapped into an \code{Array}
\end{itemize}
\vspace{5mm}

\begin{CodeBox}{rest parameters}
function sloppySum(first, ...theRest) {
  return theRest.reduce((previous, current) => {
    return previous + current;
  });
}
\end{CodeBox}
\end{frame}

%---Arguments Object ---------------------------------------------------
\begin{frame}[fragile] \frametitle{Arguments Object}
\begin{itemize}
  \item \code{arguments} is an \code{Array}-like object
  \item contains all arguments
  \item doesn't have \code{Array}'s built-in methods like \code{forEach()} and \code{map()}
  \item properties
  \begin{itemize}
    \item \code{arguments.callee}
    \item \code{arguments.caller}
    \item \code{arguments.length}
    \item \code{arguments[@@iterator]}
  \end{itemize}
\end{itemize}

\begin{CodeBox}{arguments}
function foo(a, b, c) {
  console.log(arguments[1]);
}
foo(1, 2, 3);
\end{CodeBox}
\end{frame}

%--- Arrow Function ---------------------------------------------------
\begin{frame}[fragile] \frametitle{Arrow Function}

\begin{itemize}
  \item convenient syntax
  \item is an expression
  \item creates an anonymous function
\end{itemize}
\begin{CodeBox}{syntax}
([param[, param]]) => {
   statements
}

param => expression
\end{CodeBox}
\end{frame}


%--- Arrow Function 2 ---------------------------------------------------
\begin{frame}[fragile] \frametitle{Arrow Function, examples}

\begin{CodeBox}{example of arrow functions}
let sqr = x => x*x;

let calcRectArea = (width, height) => width * height;

let pi = _ => Math.PI;

let myLogger = (msg) => {
  console.log(new Date() + ': ' + msg); 
};

let foo = (width, height) => { width * height };
\end{CodeBox}
\end{frame}

%--- Function Oriented Programming ---------------------------------------------------
\begin{frame}[fragile] \frametitle{Function Oriented Programming}

JavaScript have all features of a function oriented language.
\vspace{8mm}

\begin{CodeBox}{function oriented programming}
let list = [1, 2, 3, 4, 5];

let a = list.map(x => x + 2);
leb b = list.reduce((sum, x) => sum + x);
leb b = list.filter((x) => x % 2 === 0);
\end{CodeBox}
\end{frame}

%--- Closure ---------------------------------------------------
\begin{frame}[fragile] \frametitle{Closure}

\begin{itemize}
  \item lexical scope
  \item a closure gives you access to an outer function’s scope from an inner function
  \item closures are created every time a function is created, at function creation time
\end{itemize}
\vspace{5mm}

\begin{CodeBox}{closure}
let name = 'Per Andersson';
let foo = function() {
  name = 'anonymous';
}
console.log(name);
foo();
console.log(name);
\end{CodeBox}
\end{frame}

%--- Closure 2 ---------------------------------------------------
\begin{frame}[fragile] \frametitle{Closure}

\begin{itemize}
  \item remember, functions are values.
  \item inner functions can be returned from a function.
\end{itemize}

\vspace{4mm}

\begin{CodeBox}{closure}
let foo = function() {
  let cnt = 0;
  return _ => cnt++;
}

let idGenerator = foo();

console.log(idGenerator());
some_async_function(idGenerator);
another_async_function(idGenerator);
\end{CodeBox}
\end{frame}

