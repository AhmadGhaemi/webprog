%--- Form Validation ------------------------------------------------------------------------------
\section{Form Validation}
%--- form validation------------------------------------------------------------------------------
\begin{frame}[fragile] \frametitle{Form Validation}
\begin{itemize}
  \item user feedback is important
  \item common feedback comes from incorrectly filled forms
  \item takes a lot of time to implement
  \item html 5 introduced built in form validation
\end{itemize}
\end{frame}

%--- html 5 form validation------------------------------------------------------------------------------
\begin{frame}[fragile] \frametitle{HTML 5 Form Validation}
html form fields attributes, for example \code{<input>}:
\begin{itemize}
  \item \code{required}
  \item \code{minlength} and \code{maxlength}
  \item \code{min} and \code{max}
  \item \code{type}: \code{number}, \code{email}, \ldots
  \item \code{pattern} a regexp
\end{itemize}
\vspace{3mm}
Any error prevents form submission
\end{frame}

%--- html 5 form validation------------------------------------------------------------------------------
\begin{frame}[fragile] \frametitle{HTML 5 Form Validation}
CSS pseudo classes set by the browser
\begin{itemize}
  \item \code{:valid}
  \item \code{:invalid}
\end{itemize}
\vspace{3mm}
\begin{lstlisting}[style=htmlcssjs]
input:invalid {
  border: 2px dashed red;
}
input:invalid:required {
  background-image: linear-gradient(to right, pink, lightgreen);
}
.form-control:valid~.invalid-feedback {display: none;}
\end{lstlisting}
\end{frame}

%--- Constraint Validation API ------------------------------------------------------------------------------
\begin{frame}[fragile] \frametitle{Constraint Validation API}
adds read only properties to form input elements
\begin{itemize}
  \item \code{validationMessage}
  \item \code{validity} a \code{ValidityState} object: properties \code{rangeOverflow}, \code{valid}, et.c.
  \item \code{checkValidity()}
  \item \code{setCustomValidity(message)} makes the field invalid
\end{itemize}
\vspace{5mm}
But, you can not style the error message
\end{frame}

%--- Custom Form Validation ------------------------------------------------------------------------------
\begin{frame}[fragile] \frametitle{Custom Form Validation}
Today, form validation is based on the following principle:
\begin{itemize}
  \item use html 5 attributes to define requirements
  \item \lstinline[style=htmlcssjs]{<form novalidate>} prevents browser from displaying error messages
  \item validation is still carried out by the browser
  \item you can rely on the \code{:valid} and \code{:invalid} pseudo classes
  \item when needen, use JavaScript for custom form validation:
  \begin{enumerate}
    \item catch the \code{submit} or \code{blur} event
    \item perform custom form validation, use custom CSS classes
  \end{enumerate}
  \item error messages: use css to show or hide normal html elements
          \\ \lstinline[style=htmlcssjs]{visibility: hidden}
          \\ \lstinline[style=htmlcssjs]{<span class="my-error-class">invalid email</span>}
\end{itemize}
\vspace{5mm}
Frameworks can help you with the details.
\end{frame}

%--- Bootstrap ------------------------------------------------------------------------------
\begin{frame}[fragile] \frametitle{Bootstrap}
Bootstrap have classes for styling forms and error messages:
\begin{itemize}
  \item \lstinline[style=htmlcssjs]{<form novalidate>} to hide the browser default error messages
  \item html 5
          \\ \lstinline[style=htmlcssjs]{<input maxlength="3" type="email" class="form-control">}
          \\ \lstinline[style=htmlcssjs]{<select required class="form-select">}
          \\ \lstinline[style=htmlcssjs]{<div class="valid-feedback">well done</div>}
          \\ \lstinline[style=htmlcssjs]{<div class="invalid-feedback">not so good</div>}
  \begin{itemize}
    \item different classes for different elements
    \item must set the \lstinline[style=htmlcssjs]{.was-validated} class on a parent to show style/messages
  \end{itemize}
  \item custom validation:
  \begin{itemize}
    \item set the classes \code{.is-invalid} and \code{.is-valid} on the field element
    \item warning: \code{:valid} is set when no html 5 requirements are specified
          \\ all \lstinline[style=htmlcssjs]{valid-feedback} are shown
    \item set \lstinline[style=htmlcssjs]{.was-validated} on the form group when mixing html 5 and custom validation in the same form
  \end{itemize}
\end{itemize}
\end{frame}

%--- Bootstrap example ------------------------------------------------------------------------------
\begin{frame}[fragile] \frametitle{Bootstrap example}
\begin{lstlisting}[style=htmlcssjs]
<form novalidate>
  <div>
    <label for="field1" class="form-label">First name</label>
    <input type="text" class="form-control" id="field1" required>
    <div class="valid-feedback"> Looks good! </div>
    <div class="invalid-feedback"> no good! </div>
  </div>
  <div>
    <button class="btn btn-primary" type="submit">Submit form</button>
  </div>
</form>
\end{lstlisting}
\end{frame}
%--- Security ------------------------------------------------------------------------------
\begin{frame}[fragile] \frametitle{Security}
A note on security
\begin{itemize}
  \item client side form validation is mainly for giving users feedback
  \item a malicious user can always interrupt the network communication
  \item server can never trust data sent from the client (unless it is signed)
  \item always do server side data validation!
  \item with client side validation, server side can focus on malicious code
\end{itemize}
\end{frame}

